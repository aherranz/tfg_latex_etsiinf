%%***********************************************
%% Plantilla para TFG.
%% Escuela Técnica Superior de Ingenieros Informáticos. UPM.
%%***********************************************
%% Información requerida para completar la portada.
%%***********************************************

%% Escribe el tipo de estudios (ej. Grado)
\newcommand{\Estudios}{[ESTUDIOS, ej. Grado]}

%% Escribe el título del grado (ej. Ingeniería Informática):
\newcommand{\TituloEstudios}{[TÍTULO DE LOS ESTUDIOS, ej. Ingeniería Informática]}

% Escribe el Departamento al que pertenece el Tutor (ej. Departamento de Lenguajes y Sistemas Informáticos e Ingeniería de Software):
\newcommand{\Departamento}{[DEPARTAMENTO, ej. Departamento de Lenguajes y Sistemas Informáticos e Ingeniería de Software]}

%% Escribe Nombre y Apellidos del autor del trabajo:
\newcommand{\NombreAutor}{[NOMBRE Y APELLIDOS]}

%% Escribe Nombre y Apellidos del Tutor del trabajo:
\newcommand{\NombreTutor}{[NOMBRE Y APELLIDOS]}

%% ESCRIBE el Título del Trabajo:
\newcommand{\TituloTFG}{[Título del Trabajo, con Mayúscula en Todas las Palabras que no Sean Conectivas (Artículos, Preposiciones, Conjunciones)]}

% Escribe la fecha de lectura, en formato: mes año (ej. Enero 2021)
\newcommand{\Fecha}{[MES AÑO, ej. Enero 2021]}

% Establecer los datos con los comandos estándares de LaTeX
\title{\TituloTFG}
\author{\NombreAutor}
\date{\Fecha}
%%***********************************************

%%% Local Variables:
%%% mode: latex
%%% TeX-master: "tfg_etsiinf_plantilla"
%%% End:
