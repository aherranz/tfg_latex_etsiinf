%%***********************************************
%% Preámbulo del documento.
%%***********************************************
\usepackage[T1]{fontenc}
\usepackage[utf8]{inputenc}
\usepackage[english,spanish,es-lcroman]{babel}
\usepackage[top=3.5cm,bottom=3.5cm,right=3cm,left=3cm]{geometry}
\usepackage{bookman}
\usepackage{graphicx}
\usepackage{amsfonts,amsgen,amsmath,amssymb}
\usepackage{colortbl,longtable}
\usepackage[pdfborder={0 0 0},pdfusetitle]{hyperref}
\usepackage{url}
\usepackage{parskip} % para separar párrafos con espacio.

%%-----------------------------------------------
%% Cabeceras y pies de página
%%-----------------------------------------------
\usepackage{fancyhdr}
\pagestyle{fancy}
\fancyhf{}
\fancyhead[LO]{\leftmark}
\fancyhead[RE]{\rightmark}
\setlength{\headheight}{1.5\headheight}
\cfoot{\thepage}

\renewcommand{\chaptermark}[1]{\markboth{\textbf{#1}}{}}
\renewcommand{\sectionmark}[1]{\markright{\textbf{\thesection. #1}}}

\addto\captionsspanish{\renewcommand{\contentsname}{Tabla de contenidos}}
\setcounter{tocdepth}{3}
\setcounter{secnumdepth}{3}

%%-----------------------------------------------
%% Páginas en blanco sin cabecera:
%%-----------------------------------------------
\makeatletter
\addto\shorthandsspanish{\let\esperiod\es@period@code}

\def\clearpage{
  \ifvmode
    \ifnum \@dbltopnum =\m@ne
      \ifdim \pagetotal <\topskip
        \hbox{}
      \fi
    \fi
  \fi
  \newpage
  \thispagestyle{empty}
  \write\m@ne{}
  \vbox{}
  \penalty -\@Mi
}
\makeatother

%%-----------------------------------------------
%% Configuración relacionada con la accesibilidad del documento PDF
%% (tagged PDF).
%%-----------------------------------------------
% Comentado porque tiene algunos issues relacionados con lineno y texto formateado en headers
% \usepackage{accessibility}
