\chapter{Desarrollo}
Capítulo dedicado a describir el desarrollo del Trabajo realizado. De acuerdo con el tutor, este capítulo puede tener distintas estructuras, e incluso pueden existir varios capítulos.

Por ejemplo, para un trabajo de desarrollo clásico, este capítulo de ``Desarrollo'' podría convertirse en tres capítulos:

\begin{itemize}
\item Requisitos
\item Análisis y diseño
\item Implementación
\end{itemize}

Todos los capítulos deben empezar en una página nueva (esta plantilla ya lo hace automáticamente).

Los apartados dentro de los capítulos se numeran de forma jerárquica, pero siempre deben estar alineados al margen izquierdo. Ejemplo:

\section{Apartado 1 de capítulo 2}

\subsection{Sección 1 de apartado 1 de capítulo 2}

\subsubsection{Sub sección 1}

\subsubsection{Sub sección 2}

\subsection{Sección 2 de apartado 1 de capítulo 2}

\section{Apartado 2 de capítulo 2}

\section{Apartado 3 de capítulo 2}
