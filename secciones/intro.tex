\chapter{Introducción}
\label{ch:intro}

El capítulo de introducción debe servir para que los profesores que evalúan el trabajo puedan comprender el contexto en el que se realiza el mismo, y los objetivos que se plantean.

Esta plantilla muestra la estructura básica de la memoria final de un trabajo fin de grado (TFG), así como algunas instrucciones de formato. En el caso de una tesis de master el tutor del mismo deberá ofrecerte indicaciones más precisas.

El esquema básico de una memoria final de TFG es el siguiente:
\begin{itemize}
\item Resumen en español e inglés (máximo 2 páginas cada uno)
\item Tabla de contenidos
\item Introducción (con los objetivos del TFG)
\item Estado del arte y/o preliminares
\item Desarrollo (normalmente varios capítulos, ej. Análisis, Diseño e Implementación)
\item Análisis de impacto (puede formar parte de las conclusiones)
\item Resultados, conclusiones y trabajo futuro
\item Bibliografía (publicaciones utilizadas en el estudio y desarrollo del trabajo)
\item Anexos (opcional)
\end{itemize}

En cualquier caso, es el tutor del trabajo quien indicará a su estudiante la estructura de memoria final que mejor se ajuste al trabajo desarrollado.

Con respecto al formato, se seguirán las siguientes pautas, que se muestran en esta plantilla:
\begin{itemize}
\item \textit{Formato:} un único fichero PDF (recomendable que su tamaño no supere los $20$ Megas, no pudiendo ser superior a $50$ Megas), y opcionalmente un fichero comprimido para presentar código, ficheros de multimedia, etc. (de tamaño inferior a $10$ Megas)
\item \textit{Tamaño de papel:} DIN A4 a doble cara
\item \textit{Portada:} tal y como se recoge en las plantillas.
\item \textit{Tipo de letra para texto.} Preferiblemente ``Bookman Old Style'' $11$ puntos o equivalente, en negro.
\item \textit{Tipo de letra para código fuente:} ``Consolas'' o ``Roboto mono''.
\item \textit{Márgenes:} superior e inferior $3.5$ cm, izquierdo y derecho $3.0$ cm.
\item \textit{Superficie del texto:} $22.5$ cm de alto (aproximadamente $40$ líneas) y $15$ cm de ancho.
\item \textit{Cabecera y pies:} fuera de la superficie del texto.
\item \textit{Secciones y subsecciones:} reseñadas con numeración decimal a continuación del número del capítulo. Ej.: subsecciones 2.3.1.
\item \textit{Números de página:} siempre centrado en margen inferior, página 1 comienza en capítulo 1, todas las secciones anteriores al capítulo 1 en número romano en minúscula (i, ii, iii…).
\item \textit{Bibliografía:} según recomendaciones de la IEEE
\href{https://www.etsiinf.upm.es/docs/estudios/grado/1475_ieeecitationref.pdf}{(ver enlace)}. En principio la plantilla sigue dichas recomendaciones si metes tu bibliografía en formato bib.
\end{itemize}

\medskip

Para elaborar la memoria final del trabajo con esta plantilla, seguir los siguientes pasos:
\begin{enumerate}
\item Descargar e instalar una distribución de \LaTeX. Por ejemplo:
  \begin{itemize}
  \item MiKTeX:  \url{https://miktex.org/}
  \item TeX Live: \url{https://www.tug.org/texlive/}
  \end{itemize}
\item Descargar e instalar un editor de \LaTeX~ , por ejemplo
  Texmaker: \url{https://www.xm1math.net/texmaker/}, TeXstudio:
  \url{https://www.texstudio.org/}, o Emacs:
  \url{https://www.gnu.org/software/auctex/}.
\item Cambiar el nombre del \textbf{documento principal
    \url{`tfg_etsiinf_plantilla.tex}} para que incluya el nombre del
  alumno (ej. \url{tfg_etsiinf_LuisAmigo.tex}).
\item Modifica los datos de tu trabajo en \url{datos_tfg.tex}.
\item Observa el fichero documento principal y cómo se van incluyendo
  los contenidos de la memoria a partir de los ficheros en el
  directorio \url{secciones}.
\item Por supuesto puedes modificar la estructura de secciones
  propuestas para adecuarla al tipo de trabajo que realices (no
  olvides modificar y reordenar los \verb|\inputs| en el documento
  principal).
\item Si no sabes LaTeX te recomendamos explorar el contenido de los
  ficheros para aprender un uso básico y algunas buenas prácticas.
\item A veces es bastante común tener que compilar varias veces el
  fichero para que LaTeX calcule las referencias correctamente.
\item Si usas bibliografía, tendrás que meter entradas con formato
  bibtex en el fichero \url{secciones/biblio.bib} y usar el programa
  \url{biber} para generar la bibliografía recompilando el fichero de
  nuevo.
\end{enumerate}

\section{Algunas notas sobre el uso de \LaTeX}
\label{sec:latex}

Si es la primera vez que te enfrentas a \LaTeX\ quizás ya te hayas
percatado que se parece ``mucho'' a un lenguaje de programación:
escribes texto plano para describir el documento y luego compilas para
generar el PDF.

Lo que puedes escribir en los ficheros \LaTeX\ (ficheros \url{*.tex})
deberá seguir una sintaxis muy concreta y algunas buenas
prácticas. Las siguientes secciones entran en algunos detalles
interesantes.

\subsection{Compilación}

Cuando ejecutas el \emph{compilador} (ej. \url{pdflatex}) se genera un
PDF pero a veces las numeraciones (números de página, referencias a
seccciones o a la bibliografía) no terminan de ser correctas y
\textbf{es necesario compilar dos veces seguidas} e incluso intercalar
las compilaciones con la generación de la bibliografía (por ejemplo
con el mandato \url{biber}). En muchas ocasiones te verás ejecutando
los siguientes mandatos:

\begin{verbatim}
$ pdflatex tfg_etsiinf_LuisAmigo
$ biber tfg_etsiinf_LuisAmigo
$ pdflatex tfg_etsiinf_LuisAmigo
$ pdflatex tfg_etsiinf_LuisAmigo
\end{verbatim}

Si usas un IDE, éste suele hacer dicho trabajo por ti.

\subsection{Estructura en capítulos, secciones y párrafos}

Para estructurar tu documento basta con que incluyas marcas de dónde
empiezan los capítulos y secciones. Así\footnote{Como ya habrás podido
  observar en este mismo ejemplo.}:

\begin{verbatim}
\chapter{Título del capítulo}
\section{Título de la sección}
\end{verbatim}

Es importante observar que \textbf{los párrafos están indicados con
  dos cambios de línea}: hasta que \LaTeX\ no encuentra dos cambios de
línea no cambia de párrafo, es decir para \LaTeX\ es lo mismo un único
cambio de línea que un espacio.

Si colocas tu memoria bajo control de versiones, algo altamente
recomendado, te aconsejamos \textbf{que las líneas en el fichero \url{.tex} no
sean muy largas} para así poder ver los cambios en cada \emph{commit}
del sistema de control de versiones.

También tienes otros \emph{comandos} como \verb|\subsection|,
\verb|\subsubsection| y finalmente \verb|\paragraph|.

\subsection{Listas y enumeraciones}

Como has podido ver es fácil hacer enumeraciones o listas de ``cosas''
usando los \emph{entornos} \verb|itemize| y \verb|enumerate| (ver
más arriba en este mismo capítulo).

Puedes usar listas no numeradas:
\begin{itemize}
\item Cosa uno
\item Cosa dos
\item Cosa tres
\item Cosa cuatro
\end{itemize}

O listas numeradas:
\begin{enumerate}
\item Cosa uno
\item Cosa dos
\item Cosa tres
\item Cosa cuatro
\end{enumerate}


\subsection{Tablas}
El principal entorno para hacer tablas es \verb!tabular!. Veamos un ejemplo:

\begin{tabular}{||l|c|r|p{15em}||}
  \hline
  \hline
  \textbf{Alimento} & \textbf{Categoría} & \textbf{Kcal/100gr} & \textbf{Descripción}\\
  \hline
  \hline
  Manzana & Fruta & 52 & Es el fruto comestible de la especie Malus domestica, el manzano común. Es una fruta pomácea de forma redonda y sabor muy dulce, dependiendo de la variedad.\\
  \hline
  Repollo & Verdura & 19 & Es una planta comestible de la familia de las Brasicáceas, y una herbácea bienal, cultivada como anual, cuyas hojas lisas forman un característico cogollo compacto.\\
  \hline
  \hline
\end{tabular}

Si quieres que una tabla ocupe \textbf{más de una página} tienes que
usar el entorno \verb|longtable|. Tendrás que leer su documentación.

\subsection{Referencias}

Para poder generar \emph{etiquetas} y \emph{referencias} puedes
\textbf{y debes} usar los \emph{comandos} \verb|\label| y
\verb|\ref| como puedes ver en el capítulo \ref{ch:intro} o en la
sección \ref{sec:latex}. ¡No olvides compilar dos veces para que
\LaTeX\ regenere la numeración!.

\subsection{Tipografía y entrecomillado}

Puedes cambiar ciertas características del tipo de letra: \textrm{texto en ``roman font''}, \textbf{texto en negrita}, \emph{texto enfatizad}, \textit{texto en itálica}, \texttt{texto en teletype}, \textsc{Texto En Small Caps}. Por supuesto puedes combinar: \textbf{\textit{texto itálica en negrita}}

El entrecomillado en \LaTeX\ se realiza con estas ``comillas''. Sin
embargo, se ha incluido el paquete \url{csquotes} que permite
introducir entrecomillados sensibles al lenguaje y al contexto:
\enquote{texto entrecomillado}.


\subsection{Alineado}
El alineado de todo en \LaTeX\ es por defecto a la izquierda.

\begin{center}
  Puedes centrar casi caulquier cosa con el entorno \verb!center!
\end{center}

\hfill
Y puedes alinear a la derecha con \verb!\hfill!

\subsection{Figuras}

Puedes poner cualquier cosa dentro de una figura. Por ejemplo la
figura~\ref{fig:escudo}. LaTeX siempre intenta colocar las figuras en
el ``mejor'' sitio aunque tú le puedes orientar si la quieres
\emph{here}, \emph{top} o \emph{bottom} con \verb|[h]|, \verb|[t]| o
\verb|[b]|.

\begin{figure}[h]
  \centering
  \includegraphics[width=0.33\linewidth]{portada/escudo_etsiinf}
  \caption{El escudo de la ETSIINF}
  \label{fig:escudo}
\end{figure}

Una vez que has incluido una figura la puedes referenciar tantas veces como quieras: ver figura~\ref{fig:escudo}.

Si simplemente quieres incluir un gráfico que fluya con el texto puedes hacerlo cuando quieras como por ejemplo ahora mismo:
\begin{center}
  \includegraphics[width=0.15\linewidth]{portada/escudo_upm}
\end{center}

\section{Matemáticas}

\LaTeX\ está muy preparado para escribir fórmulas matemáticas con variables como $x$ en expresiones como esta en línea: \(\int_{a}^{b} x^2 \,dx\) o en un párrafo centrado a parte:

\begin{displaymath}
  \oint_V f(s) \,ds
\end{displaymath}

\subsection{Espaciados verticales}

Trata siempre de evitar los comando \verb|\vspace|, \verb|\newpage|, \verb|\clearpage|, \verb|\\|, etc. Casi siempre hay alternativas para hacer lo que quieres sin \emph{chapucear}.

\subsection{Citas bibligráficas}

Las citas bibliográficas se incluyen de esta forma: puede encontrar
las recomendaciones para realizar el trabajo en
\cite{recomendaciones}. Para añadir nuevas citas deberás poner las
entradas en el fichero \url{*.bib} siguiendo el formato
\emph{bibtex}~\cite{bibtex} y luego puedes referenciarla.

Esta es la cita bibliográfica de un libro \cite{stallings2006}.

\subsection{Ejemplo de ``por hacer'' (\emph{todonotes})}

Por supuesto puedes poner ``TODOS'':\todo{como este en el margen}.

\todo[inline]{O como este ``inline''}

\subsection{Ejemplo de inclusión de código fuente}

A continuación se muestra un listado de código usando el paquete \verb|listings|. En él se puede ver la función \lstinline{main()} de un programa en C que hace un \emph{hello world}.
\begin{lstlisting}[language=c]
#include <stdio.h>
// A simple Hello World
int main(){
  printf("Hello World!\n");
  return 0;
}
\end{lstlisting}

También podemos hacerlo en blanco y negro gracias a la configuración inicial:
\begin{lstlisting}[language=c,style=nocolor]
#include <stdio.h>
// A simple Hello World
int main(){
  printf("Hello World!\n");
  return 0;
}
\end{lstlisting}

También puedes usar el entorno \texttt{verbatim} y el comando \verb|\verb| pero te recomendamos usar \verb|listings| y que estudies todas sus posibilidades, no en vano eres serás ingeniera informática \verb|;)|

\subsection{Lorem Ipsum}
Lo que sigue es un lorem ipsum como ejemplo de lo que sería una
sección relativametne larga. Puedes usarlo para rellenar algo que aún
no tienes claro pero que quieres que ocupe algo de sitio para ver cómo
queda.

\lipsum
